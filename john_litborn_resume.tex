% resume.tex
% vim:set ft=tex spell:

\documentclass[10pt,letterpaper]{article}
\usepackage[letterpaper,margin=0.75in]{geometry}
\usepackage[utf8]{inputenc}
\usepackage{mdwlist}
\usepackage[T1]{fontenc}
\usepackage{textcomp}
\usepackage{tgpagella}
\usepackage{hyperref}
\pagestyle{empty}
\setlength{\tabcolsep}{0em}

% indentsection style, used for sections that aren't already in lists
% that need indentation to the level of all text in the document
\newenvironment{indentsection}[1]%
{\begin{list}{}%
	{\setlength{\leftmargin}{#1}}%
	\item[]%
}
{\end{list}}

% opposite of above; bump a section back toward the left margin
\newenvironment{unindentsection}[1]%
{\begin{list}{}%
	{\setlength{\leftmargin}{-0.5#1}}%
	\item[]%
}
{\end{list}}

% format two pieces of text, one left aligned and one right aligned
\newcommand{\headerrow}[2]
{\begin{tabular*}{\linewidth}{l@{\extracolsep{\fill}}r}
	#1 &
	#2 \\
\end{tabular*}}

% make "C++" look pretty when used in text by touching up the plus signs
\newcommand{\CPP}
{C\nolinebreak[4]\hspace{-.05em}\raisebox{.22ex}{\footnotesize\bf ++}}

% and the actual content starts here
\begin{document}

\begin{center}
    {\LARGE \textbf{John Litborn}}

    john.litborn@pm.me
    \ \textbullet \ \
    Linköping, Sweden
\end{center}

\hrule
\vspace{-0.4em}
\subsection*{Experience}

\begin{itemize}
    \parskip=0.1em

    \item
    \headerrow
        {\textbf{Education group for Programming and Programming
        Didactics}}
        {\textbf{Linköping University}}
    \\
    \textbf{Department of Computer and Information Science}
    \\
    \headerrow
        {\emph{Amanuensis, Course Assistant}}
        {\emph{2013 -- 2017}}
    \begin{itemize*}
        \item Recruited my second year at university, I taught courses in Python,
            \CPP, Ada and Matlab.
        \item Primarily introductory courses, where I held lessons with repetition
            and shared problem solving, lab assistant where I answered questions and
            helped solve problems, and correcting lab hand-ins and exams.
        \item Later on I also held more advanced courses, teaching the \CPP  standard
            library, git, object-oriented programming or unit-testing.
            I was also advisor on projects and grading documentation
            and hand-ins, and as head assistant taking on light
            administrative and managerial duties.
    \end{itemize*}

    \item
    \headerrow
        {\textbf{Division for Artificial Intelligence \& Integrated Computer Systems}}
        {\textbf{Linköping University}}
    \\
    \textbf{Department of Computer and Information Science}
    \\
    \headerrow
        {\emph{Course Developer, Software Developer}}
        {\emph{Summer of 2014 \& 2015}}
    \url{https://github.com/h00701350103/XPilot-AI\_LiU\_fork}
    \begin{itemize*}
        \item After being unhappy with a course and it's software and
            giving feedback to the professor, I was offered a summer job
            to improve it. Where I remade the assignments and improved the
            Python API for XPilot, a 2D multiplayer space shooter, that was
            used in the course.
        \item Using what I had learned in my teaching and my experience from
            the course I overhauled the structure of the assignments, changed,
            removed and added several ones and wrote better and clearer
            instructions and documentation, most of it in HTML.
        \item The XPilot-AI API in use was developed at Connecticut College,
            and after discussions with them I forked the project and begun
            modifying it to fix bugs and add features to suit our needs.
        \item My second year working on it I also started modifying the
            XPilot source code, written in C with heavy use of macros, and
            modified its network protocol to send more data so the client-side
            API for example didn't have to re-calculate the speed of objects.
    \end{itemize*}

    \item
    \headerrow
            {\textbf{Ericsson, HiQ}}
            {\textbf{Linköping}}
    \\
    \headerrow
            {\emph{IT-Consultant}}
            {\emph{2017-04 -- 2017-10}}
    \begin{itemize*}
        \item Updated 4G base-station unit tests written in Erlang to work
            in a virtualized environment. I was the git-master in my team,
            and helped the other team members when they encountered problems
            with Git or Linux.
        \item I was also responsible for updating our sections on the
            internal Wiki, and wrote python and bash scripts to simplify
            rote tasks.
    \end{itemize*}

\end{itemize}


\hrule
\vspace{-0.4em}
\subsection*{Education}

\begin{itemize}
    \parskip=0.1em

    \item
    \textbf{Linköping University}
    \\
    \headerrow
        {\emph{Faculty of Science and Engineering, Computer Engineering}}
        {\emph{2012 -- 2014, 2020-2021}}
    \begin{itemize*}
        \item Finished courses worth 120 credits, equivalent to 4 semesters of
            full-time studies.
        \item I have almost exclusively studied part-time, early on
            because of parallel work at the university, later on because of
            mental health issues.
        \item My second year I won a programming competition in a course on
            Computer Hardware and Architecture,
            which consisted of writing a sorting algorithm in microcode for
            a low-level simulated computer. My solution also beat the
            professors best implementation, and outclassed the previous student
            record.
    \end{itemize*}

\end{itemize}


\hrule
\vspace{-0.4em}
\subsection*{Core Technical Skills}

\begin{indentsection}{\parindent}
\hyphenpenalty=1000
\begin{description*}
    \item[Languages:]
        Python, C, \CPP, Ada
    \item[Tools:]
        Arch Linux, NeoVim, git, command line tools (gdb, pdb, mypy, linters)
\end{description*}
\end{indentsection}


\hrule
\vspace{-0.4em}
\subsection*{Personal Projects}

\begin{itemize}
    \parskip=0.1em

    \item
    \headerrow
        {\textbf{necro\_score\_bot}}
        {\emph{2015-present}}
    \\
    \headerrow
        {\url{https://github.com/h00701350103/necro\_score\_bot}}
        {Python 3, 1500 lines of code}
    \begin{itemize*}
        \item Lacking a good way to track scores on the leaderboards in the
            indie rhythm roguelike Crypt of the Necrodancer I decided to write
            a twitter bot that pulls leaderboards from the Steam API, parses
            them, and tweets out notable updates.
        \item Players can add a link to their twitter in their steam profile,
            which lets the bot tag them and also post less notable updates only
            to them and their followers. It auto-detects cheated or bugged
            scores and notifies the developers. It supports 16 different
            characters, 4 run types, and all permutations of 9 game modes for a
            total of over 400 leaderboards.
        \item Uses the twitter API, queries both the official and legacy steam
            API and makes raw html requests. Discord support and both
            speedrunslive and NecroLab (a community power-ranking site)
            leaderboard support was implemented but never activated. Also
            interfaces with toofz, another community site.
        \item Parses XML, json \& pickle. Multithreaded, thorough error
            handling, config file and command line flags and modularly written
            with 120 functions across 10 files.
        \item Initial development spurt during 2015, and a second spurt in 2017
            to make it work with the DLC. Has multiple small pull request from
            other community members and is kept running and maintained to this
            day by them.
    \end{itemize*}

    \item
    \headerrow
        {\textbf{Seat Exchange Bot}}
        {\emph{May-June 2019}}
    \\
    \headerrow
        {\url{https://github.com/h00701350103/seat\_exchange}}
        {Python 3, 3500 lines of code}
    \begin{itemize*}
        \item One of several small discord bots I've whipped up. This one
            during a bout of interest in the Korean game show The Genius, where
            I adapted and implemented one of the games from the show.
        \item The bot supports 40 different commands, different player counts,
            computer players, multiple simultaneous games and permissions.
        \item The code is object-oriented, easily extendable, statically
            typed (with mypy), fully adheres to PEP8 and official python coding
            standards (passes pylint with very few disabled checks) and even
            has some comments.
    \end{itemize*}

    \item
    \headerrow
        {\textbf{Home Automation}}
        {\emph{2016-present}}
    \\
    \headerrow
        {}
        {>400 actions across >30 actions}
    \begin{itemize*}
        \item Using the Android scripting application Tasker I've created a
            system for managing my life, notifying when to wake up, take
            medication \& vitamins, eat, sleep and other reminders. Logs
            everything to the calendar, as well as time-to-fall-asleep and
            sleep duration, enabling statistics.
            Also assists with daily diary taking and evaluating drugs with
            self-blinded experiments.
        \item Interfaces with my smartwatch, a Pebble Time, and widgets
            custom-created with KWGT. Planning to interface it to my self-built
            automatic roller blinds running on an Arduino.
    \end{itemize*}

\end{itemize}

\end{document}
