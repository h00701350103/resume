% resume.tex
% vim:set ft=tex spell:

\documentclass[10pt,letterpaper]{article}
\usepackage[letterpaper,margin=0.75in]{geometry}
\usepackage[utf8]{inputenc}
\usepackage{mdwlist}
\usepackage[T1]{fontenc}
\usepackage{textcomp}
\usepackage{tgpagella}
\usepackage{hyperref}
\usepackage{enumitem}
\pagestyle{empty}
\setlength{\tabcolsep}{0em}

% indentsection style, used for sections that aren't already in lists
% that need indentation to the level of all text in the document
\newenvironment{indentsection}[1]%
{\begin{list}{}%
	{\setlength{\leftmargin}{#1}}%
	\item[]%
}
{\end{list}}

% opposite of above; bump a section back toward the left margin
\newenvironment{unindentsection}[1]%
{\begin{list}{}%
	{\setlength{\leftmargin}{-0.5#1}}%
	\item[]%
}
{\end{list}}

% format two pieces of text, one left aligned and one right aligned
\newcommand{\headerrow}[2]
{\begin{tabular*}{\linewidth}{l@{\extracolsep{\fill}}r}
	#1 &
	#2 \\
\end{tabular*}}

% make "C++" look pretty when used in text by touching up the plus signs
\newcommand{\CPP}
{C\nolinebreak[4]\hspace{-.05em}\raisebox{.22ex}{\footnotesize\bf ++}}

% and the actual content starts here
\begin{document}

\begin{center}
    {\LARGE \textbf{John Litborn}}

    john.litborn@pm.me
    \ \textbullet \ \
    Linköping, Sweden
\end{center}

\hrule
\vspace{-0.4em}
\subsection*{Experience}

%\parskip=0.1em
\parindent=0em
\headerrow
{\textbf{Education Group for Programming and Programming
Didactics}}
{\textbf{Linköping University}}
\\
%\textbf{Department of Computer and Information Science}
%\\
\headerrow
{\emph{Amanuensis, Course Assistant, Head Assistant}}
{\emph{2013 -- 2017}}
\begin{itemize}[noitemsep, topsep=0pt]
    \item Taught introductory courses in Python, \CPP, Ada, and Matlab, and
        advanced courses in \CPP \ STL, Git, OO programming and unit-testing
    \item Led lessons and lab sessions, advised on projects, and graded
        hand-ins and exams
    \item Performed light administrative and managerial duties for students
        and other assistants
    \item Improved tooling and workflow used by assistants
\end{itemize}
\vspace{0.5em}
\headerrow
{\textbf{Division for Artificial Intelligence \& Integrated Computer Systems}}
{\textbf{Linköping University}}
\\
%\textbf{Department of Computer and Information Science}
%\\
\headerrow
{\emph{Course Developer, Software Developer}}
{\emph{Summer of 2014 \& 2015}}
\headerrow
{\url{https://github.com/h00701350103/XPilot-AI\_LiU\_fork}}
{Python, C, HTML}
\begin{itemize}[noitemsep, topsep=0pt]
    \item Overhauled assignments, improved instructions, and
        documentation
    \item Added features and fixed bugs in the interfaces
        between the C game client, the injection code used to run a game
        client automatically, and the Python API used by students
    \item Changed server-client netcode to send extra data, offloading and
        improving the client-side AI API
\end{itemize}
\vspace{0.5em}

\headerrow
{\textbf{Ericsson, HiQ}}
{\textbf{Linköping}}
\\
\headerrow
{\emph{IT Consultant}}
{\emph{Mar -- Oct 2017}}
\begin{itemize}[noitemsep, topsep=0pt]
    \item Updated 4G base-station unit tests written in
        Erlang to work in a virtualized environment
    \item Supported team members with core Linux and Git skills
    \item Wrote automation scripts in Python \& shell
        %\item Responsible for updating internal documentation
\end{itemize}
\vspace{0.5em}



\hrule
\vspace{-0.4em}
\subsection*{Education}
\parindent=0em
%\parskip=0.1em

\headerrow
    {\textbf{Linköping University, Faculty of Science and Engineering}}
    {\emph{2012 -- 2014, 2020 -- 2021}}
\\
\headerrow
    {\emph{Computer Engineering}}
    {120 Credits}
\begin{itemize}[noitemsep, topsep=0pt]
    \item \( 2/3 \) of credits are in programming in a diverse set of
        languages, with a focus on algorithm construction, optimization,
        low-level code, and hardware. \( 1/3 \) of credits in math courses
    \item My 2nd year I won a programming competition in a course,
        writing a sorting algorithm in microcode for
        a low-level simulated computer. It beat the
        professors record, and outclassed the standing student record
\end{itemize}
\vspace{0.5em}



\hrule
\vspace{-0.4em}
\subsection*{Core Technical Skills}

\begin{indentsection}{\parindent}
\hyphenpenalty=1000
\begin{description*}
    \item[Languages:]
        Python, \CPP, C, Ada, Shell scripting
    \item[Tools:]
        Arch Linux, NeoVim, Git, command line tools (gdb, pdb, mypy, linters)
    %\item[Secondary languages:]
    %    Java VHDL, Prolog, Assembly, GNU MathProg, JavaScript, HTML, Basic
\end{description*}
\end{indentsection}
%Meriter
%hösten 2016		Deltagare i SVT’s Genikampen, intervju
%okt 2016		1.90 poäng på högskoleprovet
%mars 2013		41/45 rätt på Mensa IQ test, 135+ IQ (99:e percentilen)
%mars 2012		2.00 poäng på högskoleprovet
%september 2012	Tilldelad Ljungbergsstipendiet på 20.000kr

\hrule
\vspace{-0.4em}
\subsection*{Personal Project Highlights}

%\parskip=0.01em

\headerrow
    {\textbf{Necro Score Bot}}
    {\emph{2015 -- present}}
\\
\headerrow
    {\url{https://github.com/h00701350103/necro\_score\_bot}}
    {Python, 1500 lines of code}
\begin{itemize}[noitemsep, topsep=0pt]
    \item Pulls leaderboard updates from the Steam API for the indie rhythm
        roguelike game Crypt of the Necrodancer, posting notable scores to
        Twitter and/or Discord
    \item Tags players and detects cheated or bugged scores, notifying the
        developers
    \item Has thorough error handling, a config file and command-line
        flags, and is modular and multithreaded
\end{itemize}

\vspace{0.5em}
\headerrow
    {\textbf{Seat Exchange Bot}}
    {\emph{May -- June 2019}}
\\
\headerrow
    {\url{https://github.com/h00701350103/seat\_exchange}}
    {Python, 3500 lines of code}
\begin{itemize}[noitemsep, topsep=0pt]
    \item Implements an adaptation of a game from a Korean game show, The
        Genius
    \item Supports 40 different commands, varying player counts,
        bot players, simultaneous games, and permissions
    \item Adheres to PEP coding standards and is Object-Oriented, modular,
        and statically typed
\end{itemize}


\end{document}
