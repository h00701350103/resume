% resume.tex
% vim:set ft=tex spell:

\documentclass[10pt,letterpaper]{article}
\usepackage[letterpaper,margin=0.75in]{geometry}
\usepackage[utf8]{inputenc}
\usepackage{mdwlist}
\usepackage[T1]{fontenc}
\usepackage{textcomp}
\usepackage{tgpagella}

\usepackage{href-ul}
\usepackage{enumitem}
\usepackage{varwidth}

% format hyperlinks
\hypersetup{
    colorlinks=true,
    urlcolor=blue
}

\pagestyle{empty}
\setlength{\tabcolsep}{0em}

% indentsection style, used for sections that aren't already in lists
% that need indentation to the level of all text in the document
\newenvironment{indentsection}[1]%
{\begin{list}{}%
    {\setlength{\leftmargin}{#1}}%
    \item[]%
}
{\end{list}}

% opposite of above; bump a section back toward the left margin
\newenvironment{unindentsection}[1]%
{\begin{list}{}%
    {\setlength{\leftmargin}{-0.5#1}}%
    \item[]%
}
{\end{list}}

% format two pieces of text, one left aligned and one right aligned
\newcommand{\headerrow}[2]
{\begin{tabular*}{\linewidth}{l@{\extracolsep{\fill}}r}
    #1 &
    #2 \\
\end{tabular*}}

% format three pieces of text, one left aligned, one center, and one right aligned
\newcommand{\threeheaderrow}[3]
{\begin{tabular*}{\linewidth}{p{110pt}l@{\extracolsep{\fill}}r}
    #1 &
    #2 &
    #3 \\
\end{tabular*}}

% make "C++" look pretty when used in text by touching up the plus signs
\newcommand{\CPP}
{C\nolinebreak[4]\hspace{-.05em}\raisebox{.22ex}{\footnotesize\bf ++}}

% fix normal plus
\newcommand{\PLUS}
{\nolinebreak[4]\hspace{-.05em}\raisebox{.22ex}{\footnotesize\bf +}}

% and the actual content starts here
\begin{document}

%%%%% HEADER %%%%%
\hyphenpenalty=1000
\begin{center}
    {\LARGE \textbf{John Litborn}}

    john.litborn@pm.me
    \ \textbullet \ \
    Linköping, Sweden
\\
\vspace{0.4em}
        %{\begin{varwidth}[t]{\linewidth}
            \emph{20\PLUS programming side projects, 15 years linux experience, scored top 0.1\% in Swedish SAT, \\ participated in The Programming Olympiads and several other competitions}
        %\end{varwidth}}

\end{center}
\hrule
\vspace{-0.4em}

%%%%% EXPERIENCE %%%%%
\subsection*{Experience}

%\parskip=0.1em
\parindent=0em
\threeheaderrow
{\emph{2013--17}}
{\textbf{Amanuensis, Course Assistant, Head Assistant}}
{\textbf{Linköping University}}
\begin{itemize}[noitemsep, topsep=1pt]
    \item Led lessons and lab sessions, advised on projects, and graded
        hand-ins and exams
    \item Taught courses in Python, \CPP, Ada, MATLAB,
        \CPP \ STL, Git, OOP,
        and unit testing
    \item Performed light administrative and managerial duties for students
        and other assistants
    \item Improved tooling and workflow used by assistants
\end{itemize}
\vspace{0.5em}
\threeheaderrow
{\emph{Summers of 2014--15}}
{\textbf{API \& Course Developer}}
{\textbf{Linköping University}}
%\\
%\headerrow
%{\url{https://github.com/jakkdl/XPilot-AI\_LiU\_fork}}
%{Python, C, HTML}
\begin{itemize}[noitemsep, topsep=1pt]
    \item Added features and fixed bugs in the interfaces
        between the C game client, the injection code used to run a game
        client automatically, and the Python API used by students
    \item Changed server--client netcode to send extra data, offloading and
        improving the client-side AI API
    \item Overhauled assignments and improved instructions and
        documentation
\end{itemize}
\vspace{0.5em}

\threeheaderrow
{\emph{Mar--Oct 2017}}
{\textbf{IT Consultant}}
{\textbf{Ericsson}}
\begin{itemize}[noitemsep, topsep=1pt]
    \item Updated 4G base-station unit tests written in
        Erlang to work in a virtualized environment
    \item Supported core Linux and Git skills for other team members
    \item Wrote automation scripts in Python \& shell
        %\item Responsible for updating internal documentation
\end{itemize}
\vspace{1em}



\hrule
\vspace{-0.4em}
\subsection*{Education}
\parindent=0em
%\parskip=0.1em

\threeheaderrow
    {\emph{2012--14, 2020--21}}
    {\textbf{Computer Engineering}}
    {\textbf{Linköping University}}
\begin{itemize}[noitemsep, topsep=1pt]
    \item  80 credits in programming courses with a focus on algorithm construction, optimization, low-level code, and hardware. Python, \CPP, C, Java, Ada, VHDL, Prolog, Assembly, GNU MathProg, Microcode.  
    \item 40 credits in math courses; statistics, logic, discrete math.
\end{itemize}
\vspace{1em}

\hrule
\vspace{-0.4em}
\subsection*{Personal Project Highlights}

%\parskip=0.01em

\headerrow
{\href{https://github.com/jakkdl/necro\_score\_bot}{\textbf{Necro Score Bot}} [1700 LoC]}
{\emph{2015--present}}
\begin{itemize}[noitemsep, topsep=1pt]
    \item Pulls leaderboard updates from the Steam API for the indie roguelike
        rhythm game \\ Crypt of the Necrodancer, posting notable scores
        to Twitter and Discord.
    \item Continously running since 2015, six github contributors.
    \item 240 \href{(https://twitter.com/necro\_score\_bot}{Twitter} followers, 13.2k tweets, {\raise.27ex\hbox{$\scriptstyle\sim$}}12.1k likes.
    \item Tags registered players and detects cheated or bugged scores, notifying the
        developers
\end{itemize}
%\href{http://www.latex-tutorial.com}{LaTeX-Tutorial}
\vspace{0.3em}
\headerrow
    {\href{https://github.com/jakkdl/seat\_exchange}{\textbf{Seat Exchange Game}} [3350 LoC]}
    {\emph{May--June 2019}}
\begin{itemize}[noitemsep, topsep=1pt]
    \item Implements an adaptation of a game from a Korean game show, \emph{The
        Genius}
    \item Supports 40 different commands, varying player counts,
        bot players, simultaneous games, and permissions
\end{itemize}
\vspace{0.2em}
Both fully linted and typed, written in Python. \\


%\vspace{1em}
\hrule
\vspace{-0.4em}
\subsection*{Merits}

2011 \textbf{Programming Olympiad}, national qualifiers: tied for 31st / 164 \\
2014 \textbf{MicroCode sorting competition} as part of a university course. Placed \textbf{1st} with an average runtime of 903.6 cycles, breaking the professors record of 948, and smashing the student record at the time. (>1100) \\
2015 \textbf{Nordic Collegiate Programming Contest} (part of ICPC): 9th/22 at our university, 108th/355 nationally.
\vspace{0.5em}

2013 \textbf{Mensa entrance test}: IQ 135\PLUS \ (top 1\%). \\
2012 \textbf{Swedish SAT}: 2.0/2.0 (top 0.1\%) \\
2016 \textbf{Swedish SAT}: 1.9/2.0 (top 0.2\%), retook it for fun with a friend

\vspace{1em}
\hrule
\vspace{1em}
%\vspace{2em}

\large{{\href{https://github.com/jakkdl/resume}{\textbf{Complete CV}}}} \normalsize{including a full list of my 20\PLUS \ programming projects is available on my GitHub.}

\end{document}
